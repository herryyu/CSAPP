%
%                                                      __----~~~~~~~~~~~------___
%                                     .  .   ~~//====......          __--~ ~~
%                     -.            \_|//     |||\\  ~~~~~~::::... /~
%                  ___-==_       _-~o~  \/    |||  \\            _/~~-
%          __---~~~.==~||\=_    -_--~/_-~|-   |\\   \\        _/~
%      _-~~     .=~    |  \\-_    '-~7  /-   /  ||    \      /
%    .~       .~       |   \\ -_    /  /-   /   ||      \   /
%   /  ____  /         |     \\ ~-_/  /|- _/   .||       \ /
%   |~~    ~~|--~~~~--_ \     ~==-/   | \~--===~~        .\
%            '         ~-|      /|    |-~\~~       __--~~
%                        |-~~-_/ |    |   ~\_   _-~            /\
%                             /  \     \__   \/~                \__
%                         _--~ _/ | .-~~____--~-/                  ~~==.
%                        ((->/~   '.|||' -_|    ~~-/ ,              . _||
%                                   -_     ~\      ~~---l__i__i__i--~~_/
%                                   _-~-__   ~)  \--______________--~~
%                                 //.-~~~-~_--~- |-------~~~~~~~~
%                                        //.-~~~--\
%                        ~~~~~~~~~~~~~~~~~~~~~~~~~~~~~~~~~~~~~~~~~~~
%  herryyu
%  2022-07-03 11:12:39
\chapter{信息的表示和处理}\label{CHP2}

我们将研究三种数字表示:
\begin{enumerate}
    \item \texttt{无符号}(unsigned)编码基于传统二进制,表示大于等于零的数字。
    \item \texttt{补码}(two's-complement)编码表示有符号整数,即可以为正或负的数字。
    \item \texttt{浮点数}(floating-point)编码表示实数的科学计数法的以2为基数的版本。
\end{enumerate}

\mksec{信息存储}
\begin{itemize}
    \item 大多数计算机使用\texttt{字节}(byte),一个8bits的块作为最小内存寻址单位。
    \item \texttt{虚拟内存}(virtual memory),各字节由一个唯一数字表示并作为其\texttt{地址}(address)
    \item 所有可能地址的集合称为\texttt{虚拟地址空间}(virtual address space)。
    \item VAS的实际实现(\mkref{CHP9}{第九章})是将DRAM、闪存、磁盘、特殊硬件与OS软件结合,为机器级程序提供一个看上去统一的字节数组。
\end{itemize}

\mksubsec{十六进制表示法}
\begin{itemize}
    \item byte的值域(二进制):$00000000_{2} \sim 11111111_{2}$ ;十进制下为$0_{10} \sim 255_{10}$ 。
    \item 二进制太冗长,十进制与位模式转换比较麻烦。替代方案为\texttt{十六进制}(hexadecimal),简写hex。
    \item hex使用 '0' $\sim$ '9' 与字符 'A' $\sim$ 'F' 表示16个值。一个hex数字对应四个二进制位。
    \item byte的值域(十六进制):$00_{16} \sim FF_{16}$
    \item C语言中hex值以\texttt{0x}/\texttt{0X}开头; 'A' $\sim$ 'F' 非大小写敏感。
    \item binary转hex:每4位一组转一个hex数字。若binary的位模式不是4的倍数,将最左边一组用0补足。
\end{itemize}

\bluetxt{练习题 2.1}详见实物书Page 26。该题涉及技巧------\underline{设值$x=2^n$,则将$x$转为hex形式的方法}:
\begin{itemize}
    \item 首先清楚$x$的binary表示为1后面跟$n$个0,且$0_{16} \Leftrightarrow 0000_{2}$。
    \item $n \div 4 = j \cdot\cdot\cdot i{\quad}(0 \geq i \geq 3)$ 。则目标hex形式为:$2^i$开头且后跟$j$个$0$ 。
    \item 比如$x=2048=2^{11}$,易知$j=2, i=3$,则$x$的hex表示为\texttt{0x800}
\end{itemize}

\bluetxt{练习题 2.2}详见实物书Page 26。该题涉及技巧------\underline{dec与hex之间的互转}:
\begin{itemize}
    \item dec数$x$转hex:将被除数$x$与除数$16$应用于短除法,结果为余数从下到上的排列表示。
    \item hex转dec:用hex的各位数乘以相应16的幂,幂的次数从$0$开始从右到左递增。
\end{itemize}

\mksubsec{字数据大小}
\begin{itemize}
    \item 每台计算机均有一个\texttt{字长}(word size),其为指针的标称大小(nominal size)。
    \item 虚拟地址以这样的一个word进行编码。
    \item 若机器的word size为$w$ bits,则其虚拟内存地址范围为$0 \sim 2^{w}-1$,程序最多访问$2^{w}$个bytes。
\end{itemize}

\textbf{基础知识:1Bytes = 8Bits ; 1KB = 1024Bytes ; 1MB = 1024KBs ; 1GB = 1024MBs ; 1TB = 1024GBs} \\
\textbf{拓展:\qquad 1GB = 2\^{}30Bytes ; 1MB = 2\^{}20Bytes ; 1KB = 2\^{}10Bytes}

\begin{itemize}
    \item 32位机器的虚拟地址空间最高能访问$2^{32}$bytes,即4GB,也即刚超过$4 \times 10^9$bytes。而64位机器为16EB,约
    $1.84 \times 10^{19}$bytes。
    \item 作为向后兼容,多数64位机器可运行32位程序,反过来则不行。
    \item \underline{C的数据类型\texttt{char}在直觉上是用来存储单个字符的,但也常被用来存储整数值}。
    \item 程序员应力图使程序具备可移植性,其中一方面便是让程序对不同数据类型的确切大小不敏感。
\end{itemize}

\mksubsec{寻址和字节顺序}
\begin{itemize}
    \item \texttt{小端法}(little endian):在内存中按从最低到最高有效字节的顺序存储对象。
    \item \texttt{大端法}(big endian):与小端法相反。
\end{itemize}

打个比方,\texttt{x}类型为\texttt{int},其值为\texttt{0x01234567},位于地址\texttt{0x100}。以下是两种表示法的区别:

\smallskip
\begin{figure}[ht]
    \centering
    \includegraphics[scale=0.35]{2-endian}
\end{figure}

\mksubsec{表示字符串}
详见实物书Page 34。

\mksubsec{表示代码}
详见实物书Page 35。

\mksubsec{布尔代数简介}
\bluetxt{网络旁注:}关于布尔代数和布尔环的更多内容

\medskip
\begin{itemize}
    \item 考虑长度为$w$的位向量上的\^{}、\&和\textasciitilde 运算,得到一种叫\texttt{布尔环}(Boolean ring)的数学形式。
    \item 整数运算中一个属性是每个值$x$均有一个\texttt{加法逆元}(additive inverse)$-x$,使得$x+(-x)=0$。Boolean ring中对应的运算为\^{},
    此时每个元素的加法逆元是其自身。
    \item 也就是说,对任何值a,a \^{} a = 0。这里的0表示全0的位向量。因此会有(a \^{} b) \^{} a = b。
\end{itemize}

\mksubsec{C语言中的位级运算}

\bluetxt{练习题 2.10}详见实物书Page 38。(该题涉及技巧------\underline{利用Boolean ring逆元运算达到值交换目的})

\medskip
位级运算常见用法是实现\texttt{掩码运算},掩码是一个位模式,表示从一个字中选出的位集合。\\
比如,掩码\texttt{0xFF}表示一个字的低位字节。比如\texttt{x=0x89ABCDEF}的运算结果为\texttt{0x000000EF}。

\medskip
\bluetxt{练习题 2.12}详见实物书Page 39。该题涉及技巧------\underline{掩码运算}

\smallskip
\bluetxt{练习题 2.13}详见实物书Page 39。该题涉及技巧------\underline{掩码运算}

\mksubsec{C语言中的逻辑运算}
\begin{itemize}
    \item C语言逻辑运算符\texttt{||}、\texttt{\&\&}、{!},分别对应命题逻辑中的OR、AND、NOT。
    \item 逻辑运算和位运算本质区别:
        \begin{enumerate}
            \item 逻辑运算中,所有非零值均视为TRUE,零自身为FALSE。
            \item 若第一个值就能确定表达式结果,则逻辑运算符会无视第二个值。比如\texttt{a\&\&5/a}不会发生零除错误,\texttt{p\&\&*p++}不会
                  导致空指针引用。
        \end{enumerate}
    \item TRUE与FALSE对应返回值分别为1与0。
\end{itemize}

\mksubsec{C语言中的移位运算}
C语言提供一组\texttt{移位}运算。

\medskip
\begin{itemize}
    \item 左移运算\texttt{x<<k}为左移$k$位,丢弃最高的$k$位,在右端补$k$个0。
    \item 右移运算分\texttt{逻辑右移}和\texttt{算术右移}:
        \begin{enumerate}
            \item 逻辑右移在左端补$k$个0.
            \item 算术右移在左端补$k$个最高有效位的值。
        \end{enumerate}
    \item 假设位模式长度为$w$,$k$值一般介于$[0, w-1]$。
\end{itemize}

C未明确定义对signed数使用哪种右移,不过大多编译器的实现为默认算术右移。对于unsigned,右移只能是逻辑的。

\mksec{整数表示}
实物书Page 42列出了CSAPP用于精确定义和描述计算机编码方式和操作整数的术语。

\mksubsec{整型数据类型}
详见实物书Page 42。

\mksubsec{无符号数的编码}
设一个整数数据类型有$w$位,以及位向量 $\vec{x}=[x_{w-1},x_{w-2},\dots,x_0]$。

\medskip
\begin{quote}
    \textbf{原理}:无符号数编码定义($\doteq$ 表示左被定义为右)
    \begin{equation}
        B2U_{w}(\vec{x})\doteq \sum_{i=0}^{w-1} x_{i}2^{i} 
    \end{equation}
\end{quote}

举个例子:$B2U_{4}([1011]) = 1 \cdot 2^3 + 0 \cdot 2^2 + 1 \cdot 2^1 + 1 \cdot 2^0 = 11$ 。\\

此时$w$位可表示值范围为:
\begin{itemize}
    \item 最小值:[0000],即整数值0。
    \item 最大值:[1111],即整数值 $UMax_w \doteq \sum_{i=0}^{w-1} 2_i = 2^w - 1$ ,令$w=4$,则$UMax_4 = B2U_{4}([1111])=15$ 。
\end{itemize}

\begin{quote}
    \textbf{原理}:无符号数编码的唯一性 \\
    每个介于$0 \sim 2^w - 1$的数均有一个唯一对应的$w$位编码。
\end{quote}